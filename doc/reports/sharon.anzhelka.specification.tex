%Cody Lewis, Luke De Ruyter, 2012
%Visit anzhelka.com for all the latest
%
% CTRL+SPACE to switch panes in Texmaker
%
% Note: blank lines indicate a new paragraph
% Note: \left( and \right) cannot break lines.

% NOTES: 
% --- Titles overlap for Engineering Effort and Societal Impacts.

% TODO: Additional material
% --- Section 2.1 still needs more content
% --- Cody take Example 1
% --- Luke take Example 2

\documentclass[english]{article}

% Packages that are being used.
\usepackage{amsmath}
\usepackage{longtable}
\usepackage{color}
%\usepackage{verbatim} % used to display code
\usepackage{listings}
\usepackage{graphicx}
\usepackage{subfigure}
\usepackage{indentfirst}
\usepackage{fancyhdr} % Fancy Header
\usepackage{rotating} % To make vertical text in tables
\usepackage[normalem]{ulem} %for strikethroughs

%\usepackage{babel}

%\usepackage{hyperref} %Must be at the end of use package but before "other settings". Used to make links of all references.




\numberwithin{equation}{section} %change the numbering to have something like 1.1 and 3.15, etc.
%Format of macro:
%\newcommand{\NAME}[ARGUMENT NUMBER (OPTIONAL)]{ stuff to include, arguments denoted #1, #2, etc. }
\newcommand{\vect}[1]{\boldsymbol{#1^2}}
\newcommand{\bigvect}[3]{\boldsymbol{#1^#2^#3}}
\newcommand{\bs}[1]{\boldsymbol{#1}}

%\addto{\captionsenglish}{\renewcommand*{\appendixname}{MyAppx}}

\definecolor{lightlightgray}{gray}{0.9}

\graphicspath{{../tests/}{../figures/}{../extra/}{../figures/editorial/}} %\graphicspath{{path/one/}{path/two/}{path/three/}}

\fancyhead[RO,RE]{\textit{Page \thepage}}
\fancyhead[CO,CE]{}
\fancyhead[LO,LE]{\textit{Software Requirements Specification for Anzhelka}}
\fancyfoot[CO,CE]{ \thepage \\ \includegraphics[width=2cm]{../extra/pheonix_small.png}}
\pagestyle{fancy}

\setcounter{tocdepth}{1} %display only sections in table of contents

\begin{document}

\pagenumbering{roman}

\lstset{
%language=C,                             % Code langugage
basicstyle=\ttfamily,                   % Code font, Examples: \footnotesize, \ttfamily
keywordstyle=\color{OliveGreen},        % Keywords font ('*' = uppercase)
commentstyle=\color{gray},              % Comments font
%numbers=left,                           % Line nums position
%numberstyle=\tiny,                      % Line-numbers fonts
%stepnumber=1,                           % Step between two line-numbers
%numbersep=5pt,                          % How far are line-numbers from code
backgroundcolor=\color{lightlightgray}, % Choose background color
frame=none,                             % A frame around the code
tabsize=4,                              % Default tab size
captionpos=b,                           % Caption-position = bottom
breaklines=true,                        % Automatic line breaking?
breakatwhitespace=false,                % Automatic breaks only at whitespace?
showspaces=false,                       % Dont make spaces visible
showtabs=false,                         % Dont make tabls visible
%columns=flexible,                       % Column format
%morekeywords={__global__, __device__},  % CUDA specific keywords
}







%\\ Version 1.0 approved \\ Prepared by Luke De Ruyter, Cody Lewis \\ University of California, Riverside \\ \date{\today}
\title{\begin{flushright}\textbf{
\rule{\textwidth}{3 pt} \\
\bigskip \bigskip
\Huge Software/Hardware \\ 
Requirements Specification \\ 
\bigskip \LARGE for \\ \bigskip
\Huge Anzhelka \\
\bigskip \bigskip \bigskip
\large Version 1.0 approved \\
\bigskip \bigskip \bigskip
Prepared by \\ Luke De Ruyter \\Cody Lewis \\
\bigskip \bigskip \bigskip
University of California, Riverside \\
\bigskip \bigskip \bigskip
\today
}\end{flushright}}
%\author{Cody Lewis \\ \texttt{srlm@anzhelka.com} \and Luke De Ruyter \\ \texttt{ilukester@anzhelka.com} }
%\date{\today}
\date{}
\author{}
\maketitle 
\newpage

%\begin{center}\includegraphics[scale=.24]{tribal_phoenix.jpg}\end{center}

%% TOC TOC TOC TOC TOC TOC TOC TOC TOC TOC TOC TOC
\renewcommand{\contentsname}{Table of Contents}
\tableofcontents
%\addcontentsline{toc}{section}{Table of Contents}
%% TOC TOC TOC TOC TOC TOC TOC TOC TOC TOC TOC TOC


%% REV REV REV REV REV REV REV REV REV REV REV REV REV
\section*{Revisions}
Current project status and files can be found at
\begin{center}
 \textbf{blog.anzhelka.com} \\
 \textbf{code.anzhelka.com} \\
\end{center}

\begin{longtable}{l | l | p{5cm} | l}
\hline
\textbf{Version} & \textbf{Date} & \textbf{Changes} & \textbf{Commiter}\\
\hline
0.01	& April 30, 2012 & Initial layout of file was created. 	& Cody \\
\hline
\end{longtable}

%% REV REV REV REV REV REV REV REV REV REV REV REV REV


\newpage
\pagenumbering{arabic}


%% S1 S1 S1 S1 S1 S1 S1 S1 S1 S1 S1 S1 S1 S1 S1 S1 S1 S1 S1 S1  

%% S1 S1 S1 S1 S1 S1 S1 S1 S1 S1 S1 S1 S1 S1 S1 S1 S1 S1 S1 S1  

%% S1 S1 S1 S1 S1 S1 S1 S1 S1 S1 S1 S1 S1 S1 S1 S1 S1 S1 S1 S1  
\section{Introduction}
\subsection{Purpose}
Anzhelka is a complete system intended for autonomous quadrotor flight. Included as a part of Anzhelka is both hardware and software. This includes the quadrotor frame, control electronics, ground station software, and the complete system documentation. Anzhelka is completely open source, and all project files are available for download. You can find in this document any instructions necessary for understanding the functionality of Anzhelka components. This includes hardware and software interfaces, features, and system requirements.
\subsection{Document Conventions}
\subsection{Intended Audience and Reading Suggestions}
This document is written for Anzhelka developers. This document is intended to refine development direction, and to bring new developers up to speed. For this document, developers include software writers, hardware designers, and system testers.

You should read this document based on your background with Anzhelka. Current developers can find the appropriate section to read. New developers should read the introduction, overall description, and system features sections. If you are a non-developer for this project, and don't intend to ever become one, you should avoid this document. Look on the Anzhelka website for something more appropriate to your needs.


\subsection{Product Scope}
Anzhelka consists of four main components: a quadrotor hardware frame, custom quadrotor software, ground station software, and detailed documentation via the Anzhelka website. Even without a degree in control systems, you can use Anzhelka components to make an autonomous quadrotor system. By using these components you can customize the functionality of the system to suit your needs, or use them directly to perform predefined commands.


\subsection{References}

\newpage
\section{Overall Description}
\subsection{Product Perspective}

\begin{figure}[h!]
  \centering
	\includegraphics[scale=.6]{elev_8_rendering.jpg}
  \caption{This is an artistic rendering for the frame.}
\end{figure}  
\textit{<Describe the context and origin of the product being specified in this SPECIFICATION. For example, state whether this product is a follow-on member of a product family, a replacement for certain existing systems, or a new, self-contained product. If the SPECIFICATION defines a component of a larger system, relate the requirements of the larger system to the functionality of this software or hardware and identify interfaces between the two. A simple diagram that shows the major components of the overall system, subsystem interconnections, and external interfaces can be helpful.>}


\subsection{Product Functions}
Anzhelka will be equipted with a Parallax Propeller Multicore Processor. It will have two separate batteries. One will provide power for the motors and servos and the other to power the electronics. There are pin headers that have been broken out for expandablitly. Some of the pin headers are broken out in servo style arangements for ease of use. The motors and servos will have voltage and current monitoring. Anzhelka has been designed so that one can easily add multiple expansion boards.

\subsection{User Classes and Characteristics}
\textit{<Identify the various user classes that you anticipate will use this product. User classes may be differentiated based on frequency of use, subset of product functions used, technical expertise, security or privilege levels, educational level, or experience. Describe the pertinent characteristics of each user class. Certain requirements may pertain only to certain user classes. Distinguish the most important user classes for this product from those who are less important to satisfy.>}
\subsubsection{User Example \#1}
\subsubsection{User Example \#2}

\begin{figure}[h!]
  \centering
	\includegraphics[scale=1]{hipster1.jpg}
  \caption{Jason Lariart}
\end{figure}  
Description goes here.
\subsection{Operating Environment}
\subsection{Design and Implementation Constraints}
\subsection{User Documentation}
\subsection{Assumptions and Dependencies}

\newpage
\section{External Interface Requirements}
\subsection{User Interfaces}
\subsection{Hardware Interfaces}
\subsection{Software or Hardware Interfaces}
\subsection{Communications Interfaces}

\newpage
\section{System Features}
\subsection{System Feature 1}
\subsubsection{Description and Priority}
\subsubsection{Stimulus/Response Sequences}
\subsubsection{Functional Requirements}
\subsection{System Feature 2 (and so on)}

\newpage
\section{Other Nonfunctional Requirements}
\subsection{Performance Requirements}
\subsection{Safety Requirements}
\subsection{Security Requirements}
\subsection{Software or Hardware Quality Attributes}
\subsection{Business Rules}

\newpage
\section{Other Requirements}
\appendix
%\gdef\thesection{Appendix \Alph{section}:}
\section{Glossary}
\section{Analysis Models}
\section{To Be Determined List}


\end{document}
